% \documentclass{easychair}
\documentclass[EPiC]{easychair}
%\documentclass[EPiCempty]{easychair}
%\documentclass[debug]{easychair}
%\documentclass[verbose]{easychair}
%\documentclass[notimes]{easychair}
%\documentclass[withtimes]{easychair}
%\documentclass[a4paper]{easychair}
%\documentclass[letterpaper]{easychair}

\usepackage{doc}

% use this if you have a long article and want to create an index
% \usepackage{makeidx}

% In order to save space or manage large tables or figures in a
% landcape-like text, you can use the rotating and pdflscape
% packages. Uncomment the desired from the below.
%
% \usepackage{rotating}
% \usepackage{pdflscape}

%\makeindex

%% Front Matter
%%
% Regular title as in the article class.
%
\title{Evaluation of Axiom Selection Techniques}
% \thanks{Other people who contributed to this document include Maria Voronkov
%   (Imperial College and EasyChair) and Graham Gough (The University of
%   Manchester).}}

% Authors are joined by \and. Their affiliations are given by \inst, which indexes
% into the list defined using \institute
%
\author{
Qinghua Liu\inst{1}
 \and
Zihao Wang\inst{2}
 \and
Zishi Wu\inst{2}
 \and
Geoff Sutcliffe\inst{2}
% \thanks{Did numerous tests and provided a lot of suggestions}
}

% Institutes for affiliations are also joined by \and,
\institute{
  System Credibility Automatic Verification Engineering Lab of Sichuan Province, Southwest Jiaotong University, China, \email{qhliu@my.swjtu.edu.cn}
\and
   University of Miami, USA, \email{zxw526@miami.edu,ry04ert39@miami.edu,geoff@cs.miami.edu}
 }

%  \authorrunning{} has to be set for the shorter version of the authors' names;
% otherwise a warning will be rendered in the running heads. When processed by
% EasyChair, this command is mandatory: a document without \authorrunning
% will be rejected by EasyChair

\authorrunning{Liu, Wang, Wu, Sutcliffe}

% \titlerunning{} has to be set to either the main title or its shorter
% version for the running heads. When processed by
% EasyChair, this command is mandatory: a document without \titlerunning
% will be rejected by EasyChair
\titlerunning{Evaluation of Axiom Selection Techniques}

\begin{document}

\maketitle
%------------------------------------------------------------------------------
\begin{abstract}
Evaluation of Axiom Selection Techniques
\end{abstract}
%------------------------------------------------------------------------------
\section{Introduction}
\label{Introduction}

GEOFF:

Intro about axiom selection. Most evaluation by running ATP, the "proofs in
the pudding". Takes time, propose Quantitative metrics based on selection.
Our methods. Evaluation vs Vampire and E.
Section on selection techniques - cuttion (eg Isabelle) and projection (eg
SInE). 

%------------------------------------------------------------------------------
\section{Selection Metrics}
\label{Metrics}

GEOFF:
Description of metrics.

%------------------------------------------------------------------------------
\section{Our Selection Techniques}
\label{Ours}

GEOFF:
Intro

QINGHUA:
Qinghua's distance

%------------------------------------------------------------------------------
\subsection{Infinity Cut}
\label{QinghuaInf}

2. Qinghua's infinity cut
%------------------------------------------------------------------------------
\subsection{A(nother) Machine Learning Approach}
\label{QinghuaML}

3. Qinghua's ML?
%------------------------------------------------------------------------------
\subsection{Our Selection Techniques}
\label{Zihao}

4. Zihaos way
%------------------------------------------------------------------------------
\subsection{Our Selection Techniques}
\label{Zishi}

5. Zishi's way

Spectral clustering is an algorithm used in Network Science research to
cluster together similar nodes in a graph. For a detailed tutorial of the 
algorithm, refer to von Luxburg \cite{vonLuxburg2007}. The algorithm consists 
of two main parts: 1) computing a feature matrix that summarizes the 
similarity between the nodes in a graph, and 2) using the feature matrix as
an input for k-means clustering in order to partition the nodes of a graph
into $k$ clusters.

Given a graph $G = (V, E)$ with a corresponding edge weight matrix $W$ such 
that $W_{ij}$ contains the measure of similarity between nodes $i$ and $j$ 
for all pairs of nodes $(i, j)$ in the graph, spectral clustering first 
constructs the graph Laplacian matrix $L = D - W$, where $D$ is the degree 
matrix of the graph. It then computes the normalized Laplacian matrix 
$L_{norm}$ using the following definition by Chung \cite{Chung1997}:
$$
L_{norm} = I - D^{-1/2} W D^{-1/2}
$$

Eigen-decomposition is performed on $L_{norm}$ to obtain its eigenvector
matrix $V$. Let $X$ denote the first $k$ columns of the matrix $V$. 
$X$ serves as a feature matrix that summarizes the similarity between the 
nodes in graph $G$. We run the k-means algorithm on $X$ in order to partition
the nodes into $k$ clusters $K_{1}, K_{2}, ..., K_{k}$.

We applied the spectral clustering algorithm to the problem of axiom 
selection as follows.

%------------------------------------------------------------------------------
\section{Evaluation Results}
\label{Results}

Section on evaluation
1. The test set(s)... Should we add tptp based set?
2. The results
3. The conclusions

Data on MPTPTP2078, Number of problems in test set, how selected (proofs,
hence already solved, but possibly with axiom selection), numbers of
different adequate subsets, average ratio nntp/all

%------------------------------------------------------------------------------
\section{Conclusion}
\label{Conclusion}

GEOFF:
1. Future correlate metrics with ptover performance (or do now!)

%------------------------------------------------------------------------------
\label{sect:bib}
\bibliographystyle{plain}
\bibliography{Bibliography}
%------------------------------------------------------------------------------
\end{document}
%------------------------------------------------------------------------------

